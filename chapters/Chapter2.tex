\chapter{NGS Data \& Quality Assessment}

\section{Initial Goals}

\begin{enumerate}
\item Understand the \textit{Sequencing by Synthesis} Process \& Data Generation \\
\item Understand what errors and artefacts lie within the data \\
\item Learn how to assess the quality of data \& make informed decisions \\
\end{enumerate}

\section{NGS Data Generation}
\begin{steps}
Before we can begin to analyse any data, it is helpful to understand how it was generated.
Whilst there are numerous platforms for generation of NGS data, today we will look at the Illumina \textit{Sequencing by Synthesis} method, which is one of the most common methods in use today.
Many of you will be familiar with the process involved, but it may be worth looking at the following 5-minute video from Illumina: \url{http://youtu.be/womKfikWlxM} 
As setting up the sound with the VMs can be tricky, it will be easier to view this from your own regular browser.
Briefly minimise the VM, open your regular browser \& please use your headphones if you brought them. \\
\end{steps}

\begin{note}
This video refers to the process \textit{tagmentation}.%\spCite{tagmentation}.
This is a relatively recent method for fragmenting \& attaching adaptors to DNA, with an alternative, more traditional method being sonication, poly-adenylation \& attachment of appropriate adaptors in separate steps.
\end{note}

\begin{information}
The important concept to note during sample preparation is that the DNA insert has multiple sequences ligated to either end.
These include 1) the \textit{sequencing primers}, 2) index or \textit{barcode} sequences, and 3) the flow-cell binding oligos.
\end{information}

\begin{questions}
Assuming each 'spot' on the flowcell is generated from a unique DNA sequence, there are two important sequencing errors that will occur during this process.
What do you think they might be? \\
\begin{answer}
1) Ligation of the wrong base during sequencing \\
2) Insertions \& deletions \\
\end{answer}

Will these errors have a more significant effect if they occur during the sequence detection stage, or during generation of the DNA clones within each cluster? \\
\begin{answer}
The earlier in the process any sequencing errors occur, the higher the degree to which they will be propagated.
If an error occurs during the very first round of amplification, before even bridge amplification, it will be propagated through an entire cluster. \\
\end{answer}
\end{questions}

\section{FASTQ File Format}
\begin{note}
As the sequences are extended during the sequencing reaction, an image is recorded which is effectively a movie or series of frames at which the addition of bases is recorded \& detected.
We mostly don't deal with these image files, but will handle data generated from these in \textit{fastq} format, which can commonly have the file suffix \textit{.fq} or \textit{.fastq}.
As these files are often very large, they will often be zipped using \texttt{gzip} or \texttt{bzip}.
Whilst we would instinctively want to unzip these files using the command \texttt{gunzip}, most NGS tools are able to work with zipped fastq files.
This can save considerable hard drive space, which is an important consideration when handling NGS datasets as the quantity of data can easily push your storage capacity to it's limit. \\
\end{note}

\begin{steps}
We should still have a terminal open from the previous section \&, if necessary, use the \texttt{cd} command to make sure you are in the \texttt{\~{}/rawData/RNASeq} directory.
The command \texttt{zcat} unzips a file \& prints the output to the terminal, or standard output (\textit{stdout}).
If we did this to these files, we would see a stream of data whizzing past in the terminal, but instead we can just pipe the output of \texttt{zcat} to the command head to view the first 10 lines of a file. \\
\begin{lstlisting}
zcat reads1.fq.gz | head
\end{lstlisting}
\end{steps}

\begin{information}
In the above command, we have used a trick commonly used in Linux systems where we have taken the output of one command (\texttt{zcat reads1.fq.gz}) and sent it to another command (\texttt{head}) by using the \textit{pipe symbol} (|).
This is literally like sticking a pipe on the end of a process \& redirecting the output to another one.
If you think of things as being like a data factory you can almost visualise it.
There are no limits to the number of commands that you can string together using this trick. \\
\end{information}

\begin{warning}
\large{Don't Panic!!!} \\
\normalsize
If you find that the terminal has become unresponsive, or you are seeing an unexpected stream of data fly past, you can abort whichever process is currently consuming the computer by entering \texttt{Ctrl-c}.
This is an instruction to the computer to `kill the current process.' \& you may be surprised at how often this comes in handy.
\end{warning}

\begin{note}
In the output from the above terminal command, we have obtained the first 10 lines of the gzipped fastq file.
This gives a clear view of the fastq file format, where each read spans four lines.
These lines are:
\begin{enumerate}
\item The read identifier
\item The sequence read
\item An alternate line for the identifier (commonly left blank as just a \texttt{+} symbol
\item The quality scores for each position along the read
\end{enumerate}
\end{note}

\subsubsection*{The read identifier}
This line begins with an \texttt{@} symbol and although there is some variability, it traditionally has several components.
Today's data have been sourced from an EBI data repository with the identifier \texttt{SRR031714}.
For the first sequence in this file, we have the full identifier \texttt{@SRR031714.1 HWI-EAS299_130MNEAAXX:2:1:785:591/1} which has the following components \\

\begin{tabular}{|p{5cm} | p{9cm} |}
  \hline
  \texttt{SRR031714.1} & The aforementioned EBI identifier \& the sequence ID within the file. As this is the first read, we have the number 1. NB: This identifier is \textbf{not} present when data is obtained directly from the machine or service provider.\\
  \hline
  \texttt{WHI-EAS299_130MNEAAXX} & The unique machine ID \\
  \hline
  \texttt{2} & The flowcell lane \\
  \hline
  \texttt{1} & The tile within the flowcell lane \\
  \hline
  \texttt{785} & The $x$-coordinate of the cluster within the tile \\
  \hline
  \texttt{591} & The $y$-coordinate of the cluster within the tile \\
  \hline
  \texttt{/1} & Indicates that this is the first read in a set of paired end reads \\
  \hline
\end{tabular}

As seen in the subsequent sections, these pieces of information can be helpful in identifying if any spatial effects have affected the quality of the reads.\\

\begin{steps}
While we are inspecting our data, have a look at the beginning of the second file.
\begin{lstlisting}
zcat reads2.fq.gz | head
\end{lstlisting}
Here you will notice that the information in the identifier is identical to the first file we inspected, with the exception that there is a \texttt{/2} at the end.
This indicates that these reads are the second set in what are known as \textit{paired-end} reads, as were introduced in the above video.
The two files will have this identical structure where the order of the sequences in one is identical to the order of the sequences in the other.
This way when they are read as a pair, they can be stepped through read-by-read \& the integrity of the data will be intact.
\end{steps}

\begin{information}
Is is also worth noting that todays reads come from a version of the Illumina \textit{casava} pipeline which is \textless 1.8, and which is a relatively common format.
For more recent reads, which have used \textgreater 1.8 of the casava pipeline, there is an additional field in the identifier which indicates whether a read would have failed an initial QC check. 
An example of this format is:
\begin{lstlisting}
@D5B4KKQ1:554:C4YHPACXX:4:1101:1084:2100 1:Y:0:
\end{lstlisting}
Note the ``Y'' in the final fields, which indicates this sequence would have failed QC.
These low-quality reads were filtered out in early versions of the pipeline and were omitted from the fastq file.
However, they are now included with this additional field indicated in the read identifier.
Inspection of the read identifiers will enable you to find out which version of the casava pipeline has been used, and whther you need to perform any additional filtering steps to remove low quality reads. 
The tool \texttt{fastq_illumina_filter} is designed to remove these reads for you \& the tool, along with usage instructions can be found at \url{http://cancan.cshl.edu/labmembers/gordon/fastq_illumina_filter/}\\
\end{information}

\subsubsection{Quality Scores}
\begin{information}
Assuming that we understand the structure of the sequence information the only other line in the fastq format that needs some introduction is the quality score information.
These are presented as single \textit{ascii} text characters for simple visual alignment with the sequence and each character corresponds to a numeric score.
In the ascii text system, each character has a representation in binary, which can also be translated to decimal numbers.\\

The first printable character is `!' which corresponds to the value 33. 
(A space is also considered printable and has the value 32, but we can ignore that)
In short, the values 33-47 are symbols like !, \", \#, \$ etc, whereas the values 48-57 are the characters 0-9.
Next are some more symbols (including @ for the value 64), with the upper case characters representing the values 65-90 \& the lower case letters representing the values 97-122.
For a full list of ascii characters see \url{http://en.wikipedia.org/wiki/ASCII#ASCII_printable_code_chart}.
\end{information}

\subsubsection{The PHRED +33/64 Scoring System}
\begin{information}
Now that we understand how to turn the quality scores from an ascii character into a numeric value, we need to know what these numbers represent.
The two main systems in common usage are PHRED +33 and PHRED +64 and for each of these coding systems we either subtract 33 or 64 from the numeric value associated with each ascii character to give us a PHRED score (or Q-value)
For example, in PHRED +33, the @ symbol corresponds to Q = 64 - 33 = 31, whereas in PHRED +64 it corresponds to Q = 64 - 64 = 0. \\
\end{information}

The following table demonstrates the comparative coding scale for the different formats: \\

\scriptsize
\texttt{SSSSSSSSSSSSSSSSSSSSSSSSSSSSSSSSSSSSSSSSS..................................................... \\
..........................XXXXXXXXXXXXXXXXXXXXXXXXXXXXXXXXXXXXXXXXXXXXXX...................... \\
...............................IIIIIIIIIIIIIIIIIIIIIIIIIIIIIIIIIIIIIIIII...................... \\
.................................\textbf{J}JJJJJJJJJJJJJJJJJJJJJJJJJJJJJJJJJJJJJJ...................... \\
LLLLLLLLLLLLLLLLLLLLLLLLLLLLLLLLLLLLLLLLLL.................................................... \\
!"\#\$\%\&'()*+,-./0123456789:;\textless =\textgreater?@ABCDEFGHIJKLMNOPQRSTUVWXYZ[\textbackslash]\^{}_`abcdefghijklmnopqrstuvwxyz\{|\}\~{}} \\
\texttt{
~~|~~~~~~~~~~~~~~~~~~~~~~~~~|~~~~|~~~~~~~~|~~~~~~~~~~~~~~~~~~~~~~~~~~~~~~|~~~~~~~~~~~~~~~~~~~~~|~\\
~33~~~~~~~~~~~~~~~~~~~~~~~~59~~~64~~~~~~~73~~~~~~~~~~~~~~~~~~~~~~~~~~~~104~~~~~~~~~~~~~~~~~~~126~\\
~ \\
 S - Sanger        Phred+33,  raw reads typically (0, 40) \\
 X - Solexa        Solexa+64, raw reads typically (-5, 40) \\
 I - Illumina 1.3+ Phred+64,  raw reads typically (0, 40) \\ 
 J - Illumina 1.5+ Phred+64,  raw reads typically (3, 40) \\
 L - Illumina 1.8+ Phred+33,  raw reads typically (0, 41) \\
}
\normalsize

\begin{questions}
In the PHRED +64 coding system, the character `@' is used.
Can you think of any potential issues this would cause when searching for it within a fastq file? \\
\begin{answer}
It is also included as the beginning of each sequence identifier.
If located at the beginning of a string of quality scores, this may be misunderstood as a sequence identifier.
This is good to keep in mind when writing custom code for searching fastq files.
\end{answer}
\end{questions}

\subsubsection{Interpretation of PHRED Scores}
\begin{note}
The quality scores are related to the probability of calling an incorrect base through the formula
\begin{equation}
  \label{eq:PHRED}
  Q = -10 log_{10} P
\end{equation}
where $P$ is the probability of calling the incorrect base. \\
\end{note}

This is more easily seen in the following table: \\
\begin{center}
\begin{tabular}[h]{|p{3cm} p{5cm} p{3cm}|}
\hline
\textbf{PHRED Score} & \textbf{Probability of Incorrect Base Call} &
\textbf{Accuracy of Base Call} \\
\hline
0 & 1 in 1 & 0\% \\
10 & 1 in 10 & 90\% \\
20 & 1 in 100 & 99\% \\
30 & 1 in 1000 & 99.9\% \\
40 & 1 in 10000 & 99.99\% \\
\hline
\end{tabular}
\end{center}

\begin{questions}
A common threshold for inclusion of a sequence is a Q score >20.
Considering the millions of sequences obtained from a flowcell, do you think that NGS is likely to be highly accurate?\\
\begin{answer}
This is really just a point for everyone to ponder.
\end{answer}
\end{questions}

\section{Using fastqc}
\begin{steps}
A common tool for checking the quality of a fastq file is the program \texttt{fastqc}.
As with all programs on the command line, we need to see how it works before we use it.
The following command will open the help file in the \texttt{less} pager which we used earlier.
To navigate through the file, use the \textless spacebar\textgreater ~to move forward, \textless \texttt{b}\textgreater ~to move back a page \& \textless \texttt{q}\textgreater ~to exit the manual. \\
\begin{lstlisting}
fastqc -h | less
\end{lstlisting}
\end{steps}

\begin{note}
Fastqc will create an html report on each file you submit, which can be opened from any web browser, such as \texttt{firefox}
As seen in the help page, \texttt{fastqc} can be run from the command line or from a graphic user interface (GUI).
Using a GUI is generally intuitive so today we will look at the command line usage, as that will give you more flexibility \& options
going forward.
Some important options for the command can be seen in the manual.\\
\end{note}
\begin{steps}
As you will see in the manual, setting the \texttt{-h} option as above will call the help page.
Look up the following options to find what they mean. \\
\begin{center}
\begin{tabular}[h]{|p{4cm}|p{8cm}|}
  \hline
  \textbf{Option} & \textbf{Usage} \\
  \hline
  -o & \\
   & \\
   \hline
  -t & \\
   & \\
  \hline
\end{tabular}
\end{center}
\end{steps}

\begin{steps}
As we have two RNA-Seq files, we will first need to create the output directory, then we can run \texttt{fastqc} using 2 threads which will ensure the files are processed in parallel.
This can be much quicker when dealing with large experiments.
\begin{lstlisting}
cd ~/rawData/RNASeq
mkdir ~/QC
fastqc -o ~/QC -t 2 reads1.fq.gz reads2.fq.gz
\end{lstlisting}
It's probably a good idea to scribble a note next to each line if you didn't understand what you did.
If you haven't seen the command \texttt{mkdir} before, check the help page 
\begin{lstlisting}
man mkdir
\end{lstlisting}
\end{steps}

\begin{steps}
The above command gave both files to fastqc, told it where to write the output (\texttt{-o \~{}/QC}) \& requested two threads (\texttt{-t 2}). 
The reports are in the html files, with all of the plots stored in the zip files. 
To look at the QC report for each file, we can use \texttt{firefox}.
\begin{lstlisting}
cd ~/QC
ls -lh
firefox reads1.fq_fastqc.html reads2.fq_fastqc.html
\end{lstlisting}
The left hand menu contains a series of click-able links to navigate through the report, with a quick guideline about each section given as a tick, cross of exclamation mark.
\end{steps}

\begin{information}
The above method will be ``driven'' form the terminal so this will appear suspended until you close firefox.
If you prefer to use a graphic interface for each folder to open a report by clicking, open this window from the drop-down menu.
\end{information}

\begin{questions}
How many sequences are there in both files?\\
\begin{answer}
  2500000 \\
\end{answer}
How long are the sequences in these files?\\
\begin{answer}
  37bp \\
\end{answer}
\end{questions}

\section{Interpreting the FASTQC Report}
\begin{note}
As we work through the QC reports we will develop a series of criteria for filtering our files.
There is usually no perfect solution, we just have to make the best decisions we can based on the information we have.
Some sections will prove more informative than others, and some will only be helpful if we are drilling deeply into our data.
Firstly we'll juts look at a selection of the plots.
We'll investigate some of the others with some `bad' data later.
\end{note}

\subsubsection*{Per Base Sequence Quality}
\begin{steps}
Both of the files should be open in firefox in separate tabs.
Perform the following steps on both files.
Click on the \texttt{Per base sequence quality} hyperlink on the left of the page \& you will see a boxplot of the QC score distributions for every position in the read.
This is the main plot that bioinformaticians will look at for making informed decisions about later stages of the analysis.
\end{steps}

\begin{questions}
What do you notice about the QC scores as you progress through the read? \\
\begin{answer}
They clearly drop off as the read extends\\
\end{answer} 
\end{questions}

We will deal with trimming the reads in a following section, but start to think about what you should do to the reads to ensure the highest quality in your final alignment \& analysis.

\paragraph{Per Tile Sequence Quality}
This section just gives a quick visualisation about any physical effects on sequence quality due to the tile within the each flowcell.
For the first file, you will notice an even breakdown in the quality of sequences near the end of the reads across all tiles.
In our second QC report, you will notice a poor quality around the 25th base in the 2nd (or 3rd) tile.
Generally, this would only be of note if drilling deeply to remove data from tiles with notable problems.

\paragraph*{Per Sequence Quality Scores}
This is just the distribution of average quality scores for each sequence.
There's not much of note for us to see here.

\paragraph{Per Base Sequence Content}
This will often show artefacts from barcode sequences or adapters early in the reads, before stabilising to show a relatively even distribution of the bases.

\paragraph{Sequence Duplication Levels}
This plot shows about what you'd expect from an RNA-Seq experiment.
There are a few duplicated sequences (rRNA, highly expressed genes etc.) and lots of unique sequences represented the diverse transcriptome.
This is only calculated on a small sample of the library for computational efficiency and is just to give a rough guide if anything unusual stands out

\paragraph{Kmer Content}
Statistically over-represented sequences can be seen here \& often they will overlap. 
In our first plot, the green \& blue sequences are the same motif, just shifted along one base.
No information is given as the source of these sequences, and you would expect to see barcode sequences or motifs that correspond to any digestion protocols here.

\subsection{Some Flawed Data}
To save running the report on some bad data, let's look at a publicly available report from some bad data.
In firefox, head to \url{http://www.bioinformatics.babraham.ac.uk/projects/fastqc/bad_sequence_fastqc.html} \& look through the report.

\paragraph{Per Base Sequence Quality}
Looking at the first plot, we can clearly see this data is not as high quality as the one we have been exploring ourselves.

\begin{questions}
Consider that the minimum sequence length required for confident mapping is >20bp.
Two approaches to this data might be to only include high quality sequences, or to trim the low quality bases from the ends and use shorter reads for downstream analysis.
What would be the consequences of either approach? \\
\begin{answer}
If we excluded low quality sequences, we would throw away a large amount of data.
Depending on the question we are asking of the data, this may render our experiment meaningless or may help us find more accurate results.
Context is everything. \\
If we trimmed the reads, we may retain a larger number of them, but more may map to non-unique locations in the reference.
Once again, context is everything.\\
\end{answer}
\end{questions}

\paragraph{Per Tile Sequence Quality}
Some physical artefacts are visible \& some tiles seem to be consistently lower quality.
Whichever approach we take to cleaning the data will more than likely account for any of these artefacts.
Sometimes it's just helpful to know where a problem has arisen.

\paragraph{Overrepresented Sequences}
Head to this section of the report \& scan down the list.
Unlike our sample data, there seem to be a lot of enriched sequences of unknown origin.
There is one hit to an Illumina adaptor sequence, so we know at least one of the contaminants in the data.
Note that some of these sequences are the same as others on the list, just shifted one or two base pairs.
A possible source of this may have been non random fragmentation.

\paragraph{Kmer Content}
\begin{questions}
Do you notice anything unusual about this plot?\\
\begin{answer}
The K-mers are present at the end of the reads.
Was this a problem with sample preparation? 
Do these map to barcodes, adaptors or primers at the other end of the reads? \\
\end{answer}
\end{questions}

\begin{information}
Interpreting the various sections of the report can take time \& experience.
A description of each of the sections is available from the \texttt{fastqc} authors at \url{http://www.bioinformatics.babraham.ac.uk/projects/fastqc/Help/}
\end{information}

\begin{bonus}
Another interesting report is available at \url{http://www.bioinformatics.babraham.ac.uk/projects/fastqc/RNA-Seq_fastqc.html}
Whilst the quality scores generally look pretty good for this one, see if you can find a point of interest in this data.
This is a good example, of why just skimming the first plot may not be such a good idea.
\end{bonus}

\begin{advanced}
In our dataset of two samples it is quite easy to think about the whole experiment \& assess the overall quality.
What about if we had 100 samples? 
Each .zip archive contains text files with the information which can easily be parsed into an overall summary. \\

Whilst this will require low-level scripting skills to perform on an experiment, we can quickly look at two of the important files today.
The overall summary in terms of PASS/FAIL is contained in the `summary.txt' file within the archive.
Open this file in the \texttt{less} pager, and once you've had a look type \texttt{q} to quit, as we have become familiar with.
\begin{lstlisting}
unzip -oc reads1.fq_fastqc.zip '*summary.txt' | less
\end{lstlisting}

The raw numbers for each of the sections are in the file fastqc_data.txt.
Page through the file, until you lose interest then quit the pager.
\begin{lstlisting}
unzip -oc reads1.fq_fastqc.zip '*fastqc_data.txt' | less
\end{lstlisting}

We can also extract any specific image file for compiling into a pdf, or find whatever we need by using these ideas.
This makes handling the data for a large experiment much simpler.
There are plenty of hints online for how to write a \textit{shell script}, or alternatively, attend one of our scripting workshops.
\end{advanced}

\section{Further Reading}
An excellent article which deals with some of the issues around data quality is:

Zhou, X and Rokas, A. (2014). \textit{Prevention, diagnosis and treatment of high-throughput sequencing data pathologies.} Molecular Ecology 23, 1679-1700.

This has been included on your VM as the file QC.pdf \& contains many examples of good data and low quality data, as well as a detailed discussion.
If you feel like you are running ahead of schedule, or if you finish early it may be a good opportunity to download \& read through the article.
The workflow given at the end may be particularly useful.